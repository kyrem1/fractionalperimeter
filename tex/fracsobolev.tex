%! TEX root = ../main.tex
\documentclass[../main.tex]{subfiles}


\begin{document}

\section{Sobolev Space Preliminaries}


The theory behind fractional perimeter is written in the language of fractional Sobolev spaces. As such, we will motivate and define these function spaces as well as discuss their most relevant properties. 

% TODO Talk about including illuminating proofs in the appendix

\subsection{Fractional Sobolev Spaces}

\begin{definition}\label{fracnorms}
    Fix $ 1\leq p <+\infty $ and let $ s\in (0,1) $ be a fractional exponent. For $ u\in L^{p}(\Omega) $, define the \textit{Gagliardo (semi)norm} of $ u $ to be the quantity
    \[
        [u]_{W^{s,p}(\Omega)} := \lr{\int_{\Omega}\int_{\Omega} \frac{|u(x)-u(y)|^{p}}{|x-y|^{n+sp}}\dd{x}\dd{y}}^{\frac{1}{p}}.
    \]
    and define the \textit{fractional Sobolev space} $ W^{s,p}(\Omega):= \{u\in L^{p}(\Omega):[u]_{W^{s,p}(\Omega)}<+\infty\} $. This is a Banach space with the natural norm
    \[
        \norm{u}_{W^{s,p}(\Omega)} := \norm{u}_{L^{p}(\Omega)}+[u]_{W^{s,p}(\Omega)}.
    \]
    We remark that $ C^{\infty}_{c}(\Omega)\sub W^{s,p}(\Omega) $ and we write $ W_{0}^{s,p}(\Omega) $ for the closure of $ C^{\infty}_{c}(\Omega) $ inside $ W^{s,p}(\Omega) $. It is a fact that when $ \Omega = \R^{n} $, these two spaces are equal; however, this is not necessarily true for general $ \Omega $.
    %TODO talk about true for sufficiently regular $ \Omega $ when sp\leq 1 
    %TODO insert counterexample

    In the case $ p=2 $, $ W^{s,2}(\Omega) $ is in fact a Hilbert space with inner product given by 
    \[
        \inp{u}{v}_{H^{s}(\Omega)} := \int_{\Omega} u(x)v(x)\dd{x} + \int_{\Omega}\int_{\Omega} \frac{(u(x)-u(y))(v(x)-v(y))}{|x-y|^{n+2s}} \dd{x}\dd{y}.
    \]

    In a somewhat precise sense (real interpolation), the fractional sobolev spaces $ W^{s,p}(\Omega) $ are intermediary spaces between $ L^{p}(\Omega) $ and the classical Sobolev space $ W^{1,p}(\Omega) $.
\end{definition}



Althought these spaces and (semi)norms seem somewhat natural from the viewpoint of being an analogue of the H\"{o}lder condition for $ L^{p} $ spaces instead of $ L^{\infty} $, when presented as above they are ultimately quite artificial. Where would one find such spaces appearing in nature?

If one takes for granted that integer sobolev spaces ``appear in nature,'' then the answer to the previous question is that \textit{fractional sobolev spaces appear as the correct image of the trace operator} (see Appendix A for background on the trace operator).
%TODO insert trace theorem as evidence of W^{s,2} appearing in nature
%TODO maybe do different trace theorem like the one for $ W^1,p $

%TODO DEFINE W_0
\begin{proposition}
    Suppose $ \Omega\sub \R^{n} $ is a nice domain (see definition \ref{extndomain}) and $ k\in \N $. Then there is a split exact sequence of Hilbert spaces
\[\begin{tikzcd}
    0 & {W_0^{k,2}(\Omega)} & {W^{k,2}(\Omega)} & {W^{k-\frac{1}{2},2}(\partial\Omega)} & 0
	\arrow[from=1-1, to=1-2]
	\arrow[hook, from=1-2, to=1-3]
	\arrow["T", two heads, from=1-3, to=1-4]
	\arrow[from=1-4, to=1-5]
\end{tikzcd}\]
    where $ T: W^{k,2}(\Omega) \to W^{k-\frac{1}{2},2}(\partial\Omega)$ is the trace operator.
    
\end{proposition}

\subsection{Extension Domains}

\begin{definition}\label{extndomain}
    An open set $ \Omega\sub \R^{n} $ is an \textit{extension domain for $ W^{s,p} $} if  there exists a constant $ C = C(s,p,n,\Omega) > 0 $ such that for every $ u\in W^{s,p}(\Omega) $, there exists a $ \widetilde{u}\in W^{s,p}(\R^{n}) $ such that 
    \[
        \widetilde{u}\vert_{\Omega} \equiv u \quad\text{and }\quad \norm{\widetilde{u}}_{W^{s,p}(\R^{n}}\leq C \norm{u}_{W^{s,p}(\Omega)}.
    \]
\end{definition}

We remark that any bounded open set with Lipschitz boundary is an extension domain. See Hitchhiker`s Guide (\cite{hitchhiker}) for details. 

Armed with this definition, we explore the properties of $ BV(\Omega) $ when $ \Omega $ is an extension domain.


\begin{proposition}\label{BVinWs}
    Suppose that $ \Omega\sub \R^{n} $ is an extension domain. Then for $ s\in (0,1) $ we have a continuous embedding $ BV(\Omega)\hookrightarrow W^{s,1}(\Omega) $.
\end{proposition}

\begin{proof}
    We follow \textcite{lobardini:2019}. 
\end{proof}


\subsection{The Fractional Laplacian}

\begin{definition}
    Fix $ s\in (0,1) $. We define the fractional Laplacian $ (-\Delta)^{s}:\S\to L^{2}(\R^{n}) $ as a Fourier multiplier given by
    \[
        (-\Delta)^{s}u = \mathscr{F}^{-1} (| \xi|^{2s}( \mathscr{F}u)).
    \]
\end{definition}

\subsection{$ H^{s} $: An Alternative approach to fractional Sobolev spaces using $ \mathscr{F} $}

\begin{definition}
    Let $ s\in (0,1) $. Consider the space
    \[
         H^{s}(\R^{n}) := \{ u\in L^{2}(\R^{n}): \int_{\R^{n}}(1+| \xi|^{2})^{s}|\mathscr{F}u|^{2}\dd{\xi} < +\infty\} 
    \]
    equipped with the norm
    \[
        [u]_{H^{s}}^{2} = \int_{\R^{n}}(1+| \xi|^{2})^{s}|\mathscr{F}u|^{2}\dd{\xi} 
    \]
\end{definition}


\end{document}
