%! TEX root = ../main.tex
\documentclass[../main.tex]{subfiles}


\begin{document}

\section{Fractional $s$-Perimeter}\label{sec:sperim}
First introduced by Caffarelli in 2011 \cite{caffarelli:nonlocal, caffarelli:limit}, the study of the fractional perimeter (and more generally nonlocal perimeters) has devloped into a highly active area of research. We mention the 2019 monograph by Maz\'on \cite{mazon:nonlocal} for an overview of the development of the field. 


%\subsection{Why study the Asymptotics}
%
%\begin{fact}
%    For any measurable subset $ E\sub \R^{n} $ of finite positive measure, the characteristic function $ \chi_{E} $ is not an element of $ H^{\frac{1}{2}}(\R^{n}) $.
%\end{fact}
%
%\begin{proof}
%    By assumption $ \chi_{E}\in L^{2}(\R^{n})$, so it suffices to show that $ \chi_{E} $ is not in the corresponding homogenous sobolev space $ \dot H^{\frac{1}{2}}(\R^{n}) $. \\  
%
%
%    Let $ B $ be the ball centered at $ 0 $ with the same (finite) measure as $ E $, i.e. the symmetric decreasing rearrangement of the set $ E $. (INSERT EXPOSITION ABOUT THIS)$ \chi_{E}^{*} = \chi^{B} $ and we have that 
%    \[
%        \norm{\chi_{E}}_{\dot H^{\frac{1}{2}}(\R^{n})} \geq \norm{\chi_{E}^{*}}_{\dot H^{\frac{1}{2}}(\R^{n})} = \norm{\chi_{B}}_{\dot H^{\frac{1}{2}}(\R^{n})}
%    \]
%    \begin{align*}
%        \norm{\chi_{B}}_{\dot H^{\frac{1}{2}}(\R^{n})} = \int_{\R^{n}} (1+| \xi|^{2})^{\frac{1}{2}}| \widehat{\xi_{B}}(\xi)|^{2} \dd{\xi}
%    \end{align*}
%
%    To estimate this ``symmetrized'' Gagliardo seminorm, we must compute the fourier transform of the characteristic function of a ball. Let $ R $ be the radius of $ B $ and note that, as $ \chi_{B} $ is rotationally symmetric, so is its Fourier transform. Hence we evaluate at the point $ \xi = (0,0,\ldots, 0, \rho) $ using polar coordinates
%    \begin{align*}
%        \widehat{\xi_{B_{R}(0)}}(\xi) &= \frac{1}{(2\pi)^{n/2}}\int_{B_{R}(0)} e^{-i x\cdot \xi} \dd{x} = \frac{1}{(2\pi)^{n/2}}\int_{B_{R}(0)} e^{-i \rho x_{n} } \dd{x} \\
%        &=  \frac{1}{(2\pi)^{n/2}}\int_{S^{n-1}}\int_0^{R} e^{-i \rho x_{n} } \dd{x} 
%    \end{align*}
%
%    \begin{align*}
%        \widehat{\chi_{B_{R}(0)}}(\xi) &= \frac{1}{(2\pi)^{n/2}}\int_{\R^{n}} \chi_{B_{R}(0)}(x) e^{-i x\cdot \xi} \dd{x} = \frac{1}{(2\pi)^{n/2}}\int_{\R^{n}} \chi_{B_{R}(0)}(x) e^{-i \rho x_{n}} \dd{x}  
%    \end{align*}
%\end{proof}
%
%\begin{fact}
%    For any measurable subset $ E\sub \R^{n} $ of finite positive measure, the characteristic function $ \chi_{E} $ is not an element of $ W^{1,1}(\R^{n})$.
%\end{fact}
%
%
%
%% ToDO include stuff about symmetric rearrangement
%
%
%
%
%
%%TODO talk about where it was introduced and its usage
\begin{definition}\label{def:sperim}
    Let $ \Omega\sub \R^{n} $ be a bounded domain or all of $ \R^{n} $, $ s\in (0,1) $, and $ E\sub \R^{n} $ measurable. Following Figure \ref{fig:interactions} from the introduction, the \textit{fractional $ s $-perimeter of $ E $ in $ \Omega $} is the quantity
    \begin{align*}
        P_{s}(E, \Omega) := L_{s}(E\cap \Omega, E^{c}\cap \Omega) + L_{s}(E\cap \Omega, E^{c}\cap \Omega^{c}) + L(E\cap \Omega^{c}, E^{c} \cap \Omega)
    \end{align*}
    where $ L(A,B) $ denotes the following interaction energy integral
    \[
        L(A,B) := \int_{A}\int_{B}\frac{1}{|x-y|^{n+s}}\dd{x}\dd{y} \text{ for all } A,B\sub \R^{n}
    \]
    with the convention that $ L(A,B) = 0 $ if either $ A $ or $ B $ is empty. When $ \Omega =\R^{n} $, we write $ P_{s}(E):= P_{s}(E,\R^{n}) $.
\end{definition}
For the reader's convenience moving forward, we record a simple characteristic function computation
\begin{align*}
    | \chi_{E}(x) - \chi_{E}(y)| &= | \chi_{E}(x) - \chi_{E}(y)|^{2} = \chi_{E}(x)-\chi_{E}(x) \chi_{E}(y) +\chi_{E}(y) -\chi_{E}(x) \chi_{E}(y)\\
    &= \chi_{E}(x)(1- \chi_{E}(y)) + \chi_{E}(y)(1-\chi_{E}(x)) = \chi_{E}(x) \chi_{E^{c}}(y) + \chi_{E^{c}}(x) \chi_{E}(y).
\end{align*}
Now, to place the functional $ P_{s} $ in the context of Section \ref{sec:sobolev}, consider the following reformulation. 
\begin{align*}
    P_{s}(E,\Omega)&= L(E\cap \Omega, E^{c} \cap \Omega) + L(E^{c}\cap \Omega, E\cap \Omega^{c}) + L(E\cap \Omega, E^{c}\cap \Omega^{c}) \\
    &= \int_{\Omega}\int_{\Omega}\frac{\chi_{E}(x) \chi_{E^{c}}(y)}{|x-y|^{n+s}}\dd{x}\dd{y} + \lr{\int_{\R^{n}\setminus\Omega}\int_{\Omega}\frac{\chi_{E^{c}}(x) \chi_{E}(y)}{|x-y|^{n+s}}\dd{x}\dd{y} + \int_{\Omega}\int_{\R^{n}\setminus\Omega}\frac{\chi_{E^{c}}(x) \chi_{E}(y)}{|x-y|^{n+s}}\dd{x}\dd{y} } \\
    &\overset{\text{Fubini}}{=} \frac{1}{2}\int_{\Omega}\int_{\Omega} \frac{\chi_{E}(x) \chi_{E^{c}}(y) + \chi_{E^{c}}(x) \chi_{E}(y)}{|x-y|^{n+s}}\dd{x}\dd{y} + \int_{\Omega}\int_{\R^{n}\setminus \Omega} \frac{\chi_{E}(x) \chi_{E^{c}}(y) + \chi_{E^{c}}(x) \chi_{E}(y)}{|x-y|^{n+s}}\dd{x}\dd{y}\\
    &= \frac{1}{2}[\chi_{E}]_{W^{s,1}(\Omega)} + \int_{\Omega}\int_{\R^{n}\setminus \Omega} \frac{| \chi_{E}(x) - \chi_{E}(y)|}{|x-y|^{n+s}}\dd{x}\dd{y} 
\end{align*}


\begin{definition}
    With the same setup as Definition \ref{def:sperim}, we define the local and nonlocal energy terms by
    \[
        P_{s}^{L}(E,\Omega) := \frac{1}{2}[\chi_{E}]_{W^{s,1}(\Omega)} \quad \quad P_{s}^{NL}(E,\Omega) := \int_{\Omega}\int_{\R^{n}\setminus \Omega} \frac{| \chi_{E}(x) - \chi_{E}(y)|}{|x-y|^{n+s}}\dd{x}\dd{y} 
    \]
    Hence we have a decomposition $ P_{s}(E,\Omega) = P_{s}^{L}(E,\Omega) + P_{s}^{NL}(E,\Omega) $ of the fractional perimeter into local and nonlocal energy terms.
\end{definition}

\end{document}
