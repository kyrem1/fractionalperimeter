%! TEX root = ../main.tex
\documentclass[../main.tex]{subfiles}


\begin{document}

\section{Bounded Variation and Cacciopoli Sets}\label{sec:bv}

\subsection{The Spaces $ BV $ and $ BV_{loc} $}


\begin{figure}[H]
\includesvg{figures/indicator.svg}
\end{figure}

Our story begins with the classical Gauss-Green formula. Recall, when $ \Omega\sub \R^{n} $ is open, bounded with $ C^{1} $ boundary, we have for every smooth $ \R^{n} $-valued function $ \Phi $ that
\begin{equation}\label{gaussgreen}
    \int_{\Omega} \div{\Phi} \dd{\H^{n}} = \int_{\partial \Omega} \Phi\cdot \nu \dd{\H^{n-1}},
\end{equation}
where $ \nu\in C^{1}(\partial \Omega, \R^{n})  $ is the outward pointing unit normal vector field on $ \partial \Omega $.



De Giorgi`s program in the 1950s revolved around trying to make sense of \eqref{gaussgreen} when the topological boundary of $ \Omega $ is no longer smooth. The following section will outline his work in this area. De Giorgi begins with the following idea. Suppose that now we have a set $ E $ which is not necessarily smooth. As the characteristic function $ \chi_{E} $ is locally integrable in $ \R^{n} $, we can consider $ \chi_{E} $ as a distribution via integration against $ \chi_{E} $. Thus, it makes sense to talk about the distributional derivatives $ D_{i}\chi_{E} $ of $ \chi_{E} $. 

Assume that each distribution $ D_{i} \chi_{E} $ is in fact represented by some Radon measure, which by abuse of notation we also write $ D_{i} \chi_{E} $. Then the distributional gradient is in fact represented by the vector-valued Radon measure $ D\chi_{E} = (D_{1} \chi_{E}, \ldots, D_{n}\chi_{E}) $.

Following this discussion, we then compute for smooth vector fields $ \Phi = ( \Phi^{1}, \ldots, \Phi^{n})\in [\D(\R^{n})]^{n} $,
\begin{align*}
    \int_{\R^{n}} \chi_{E}\div{\Phi} \dd{x} &= \sum_{i=1}^{n}\int_{\R^{n}} \chi_{E} \pdv{\Phi^{i}}{x_{i}} \dd{x} = \sum_{i=1}^{n} -\inp{D_{i} \chi_{E}}{\Phi^{i}} \\
    &= \sum_{i=1}^{n} -\int_{\R^{n}} \Phi^{i} \dd{D_{i} \chi_{E}}= - \int_{\R^{n}} \Phi\cdot \dd{D \chi_{E}}.
\end{align*}

Now let $ |D \chi_{E}| $ denote the total variation measure of $ D \chi_{E} $. Then by Radon-Nikodym, there exists a $ \mu $-measurable function $ \sigma:\R^{n}\to \R^{n} $ with $ | \sigma | = 1 $ $ |D \chi_{E}| $-a.e. such that 
\[
    \dd{D \chi_{E}} = \sigma\cdot \dd{|D \chi_{E}|}.
\]
Then the above equation becomes
\[
    \int_{E} \div{\Phi} \dd{x} = \int_{\R^{n}} \chi_{E} \div{\Phi} \dd{x} = -\int_{\R^{n}} \Phi\cdot \sigma \dd{|D \chi_{E}|}.
\]

The latter integral being over all of $ \R^{n} $ is quite unsatisfactory, so it is a natural question to ask where the measure  $ |D \chi_{E}| $ is supported. Our intuition would point to $ |D \chi_{E}| $ being supported on the boundary, since the quantity is some kind of ``gradient'' of $ \chi_{E} $ which is constant everywhere else. Moreover, as there are vertical jumps on $ \partial E $, it would make sense for the ``gradient'' at these jumps to be vertical of ``infinite length'' (that is, a dirac delta at every point on the boundary). Of course, this is just intuition and not rigorous mathematics, but it is not bad intuition (see Section \ref{sec:perimeter}).

\begin{claim}
    $ \supp(D \chi_{E}) \sub \partial E $.
\end{claim}

\begin{proof}
    Suppose $ z\in \R^{n}\setminus \partial E $. Then there is some open, bounded neighborhood $ U $ of $ z $ with smooth boundary such that $ U\sub (\R^{n}\setminus \partial E)^{o} $. Thus $ U $ is either in the interior of $ E $ or the interior of $ \R^{n}\setminus E $. 

    If $ U\sub (\R^{n}\setminus E)^{o} $, then for $ \Phi\in [\D(\R^{n})]^{n} $ with $ \supp(\Phi)\sub U $, we have
    \[ 
        \int_{\R^{n}} \Phi\cdot \dd{D \chi_{E}} = -\int_{\R^{n}}\chi_{E} \div{\Phi} \dd{x} = -\int_{U} \chi_{E}\div{\Phi}\dd{x} = 0.
    \]

    If $ U\sub E^{o} $, then for smooth vector fields supported within $ U $ we have by \eqref{gaussgreen} that
    \[
        \int_{\R^{n}} \Phi\cdot \dd{D \chi_{E}} = -\int_{\R^{n}}\chi_{E} \div{\Phi} \dd{x} = -\int_{U} \div{\Phi}\dd{x} = -\int_{\partial U} \Phi\cdot \nu_{U} \dd{\H^{n-1}} = 0.
    \]
    By density, these formulae actually hold for all $ \Phi\in C^{1}_{c} (\R^{n},\R^{n}) $ with $ \supp(\Phi)\sub U $. Hence, $ D\chi_{E}\vert_{U} \equiv 0 $, so $ z\not\in \supp(D\chi_{E}) $.
\end{proof}


Hence, setting $ \nu = -\sigma $ (so $ \nu $ is like a generalized outward normal vector field), we recover a statement which looks like Gauss Green:
\[
    \int_{E} \div{\Phi} \dd{\H^{n}} = \int_{\partial E} \Phi\cdot \nu\dd{|D \chi_{E}|}
\]
We obtained such a formula by considering sets $ E $ such that the distributional gradient of $ \chi_{E} $  is represented by a vector-valued Radon measure. More generally, we can consider integrable (or locally integrable) functions $ f $ whose distributional gradient is represented by a vector-valued Radon measure. This line of thought leads to the notion of \textit{functions of bounded variation}.

\begin{definition}\label{def:BV}
    Let $ \Omega\sub \R^{n} $ be an open set and $ u\in L^{1}(\Omega) $. We say that $ u $ is of \textit{bounded variation in $\Omega$}, written $ u\in BV(\Omega) $, if the the distributional derivatives $ D_{i}u $ are represented by finite Radon measures. We then form the vector-valued measure $ Du:=(D_{1}u,\ldots, D_{n}u) $. Given $ u\in BV(\Omega) $, the $ BV(\Omega) $-seminorm of $ u $ is the quantity 
    \[
        [u]_{BV(\Omega)} := |Du|(\Omega).
    \]
    We note that $ BV(\Omega) $ is in fact a Banach space with norm 
    \[
        \norm{u}_{BV(\Omega)} := \norm{u}_{L^{1}(\Omega)} + [u]_{BV(\Omega)}.
    \]
\end{definition}

\begin{example}
    If $ u\in W^{1,1}(\Omega) $, then $ u\in BV(\Omega) $ and $ dD \chi_{E} = Du \,d \mathcal{L}^{n} $. This follows by definition of the distributional derivative.
\end{example}

We would like a quantitative way to test whether a given function $ u\in L^{1}(\Omega) $ is of bounded variation. The above definition of $ BV(\Omega) $-seminorm is not satisfactory as it requires a function to have distributional derivative represented by a measure to be intelligible. \textit{A priori}, the distributional gradient is just a continuous linear functional on the locally convex space $ C^{\infty}_{c}(\Omega,\R^{n}) $. It is not true that such a functional is in general given by a integration against an $ \R^{n} $-valued Radon measure.

Thankfully, the Riesz(-Markov-Kakutani) representation theorem gives us conditions upon which which linear functionals are given by integration against measures. We state the vector-valued version of the theorem, as this form may not be as familiar to the reader. 
\begin{theorem}[Riesz Representation Theorem]\label{rrt}
    Let $ X $ be a locally compact, Hausdorff space. Suppose $ L: C_{0}(X,\R^{n})\to \R $ is a linear functional which is bounded, i.e. 
    \[
        \norm{L}:= \sup\{L(\phi) : \phi\in C_{0}(X,\R^{n}), | \phi|\leq 1\} < +\infty.
    \]
    Then there is a unique finite $ \R^{n} $-valued Radon measure $ \mu = (\mu_{1},\ldots, \mu_{n}) $ on $ X $ such that 
    \[
        L(u) = \int_{X} \phi\cdot \dd{\mu} = \sum_{i=1}^{n}\int_{X} \phi_{i} \dd{\mu_{i}}.
    \]
    Moreover, we have $ \norm{L} = | \mu|(X) $.
\end{theorem}
Motivated by the norm-boundedness condition in the above theorem, we make the following definition.
\begin{definition}Given a function $ u\in L^{1}(\Omega) $, define the \textit{total variation of $ u $} to be the quantity
    \[
        V(u,\Omega) = \sup\left\{ \int_{\Omega} u\div{\phi}\dd{x} : \phi\in C_{c}^{1}(\Omega,\R^{n}), | \phi| \leq 1 \right\}.
    \]
\end{definition}
Note that this is simply saying that the  distributional gradient is bounded when considered as a linear functional on $ C_{c}^{1}(\Omega,\R^{n}) $. This does not automatically satisfy the conditions of Theorem \ref{rrt} as $ L $ is not defined on all of $ C_{0}(\Omega,\R^{n}) $; however, a density argument rectifies this problem.

\begin{proposition}[Characterization of $ BV $]\label{bvchar}
    Suppose $ u\in L^{1}(\Omega) $ has finite total variation. Then there exists a finite Radon measure $ \mu $ on $ \Omega $ and a $ \mu $-measurable $ \sigma:\Omega\to \R^{n} $ with $ | \sigma| = 1$ $ \mu $-a.e. and
    \[
        \int_{\Omega} u\div{\phi}\dd{x} = -\int_{\Omega} \phi\cdot \sigma \dd{\mu} \text{ for all } \phi\in C_{c}^{1}(\Omega,\R^{n}).
    \]
    Moreover, $ V(u,\Omega) = | \mu|(\Omega) $.
\end{proposition}

\begin{proof}
    Define a linear functional $ L:C_{c}^{1}(\Omega,\R^{n})\to \R $ by $ L(\phi):= -\int_{\Omega}u\div{\phi} \dd{x} $. The boundedness of $ L $ combined with linearity implies
    \[
        |L(\phi)|\leq V(u,\Omega)\norm{\phi}_{\infty}\text{ for all } \phi\in C_{c}^{1}(\Omega,\R^{n}).
    \]

    Fix $ \phi\in C_{c}(\Omega,\R^{n}) $. Since $ \phi $ is continuous and compactly supported, we can choose a sequence $ (\phi_{i})_{i} $ in $ C_{c}(\Omega,\R^{n}) $ such that $ \supp(\phi)\sub \supp(\phi_{i}) $ and $ \phi_{i}\to \phi $ uniformly in a fixed compactly contained neighborhood of $ \supp(\phi) $. 

    Define an extension $ \widetilde{L}:C_{c}(\Omega,\R^{n})\to \R $ of $ L $ by $ \widetilde{L}(\phi) = \lim_{k\to\infty}L(\phi_{k}) $, which exists and is well-defined by the above inequality. As $ C_{c}(\Omega, \R^{n}) $ is dense in $ C_{0}(\Omega, \R^{n}) $, continuity guarantees a unique continuous linear extension to the whole of $ C_{0}(\Omega,\R^{n}) $. Applying the Riesz Representation Theorem to this extension gives the conclusion.
\end{proof}
The above proposition allows us to extend our definition of $ BV(\Omega) $-seminorm to all of $ L^{1}(\Omega) $.
\begin{definition}
    Given a function $ u\in L^{1}(\Omega) $, define the \textit{$ BV(\Omega) $-seminorm of $ u $} to be the quantity $ [u]_{BV(\Omega)} := V(u,\Omega) $. Note that for $ u\in L^{1(\Omega)} $, we have $ u\in BV(\Omega)\iff [u]_{BV(\Omega)} <+\infty $.
\end{definition}

\begin{remark}
    Although not used in this paper, we remark that there are local versions of both Definition \ref{def:BV} and Proposition \ref{bvchar}.
\end{remark}


%\begin{definition}    Similarly, given $ u\in L^{1}_{loc}(\Omega) $ and $ U\Subset \Omega $, define the \textit{local variation of $ u $ in $ U $} by
%    \[
%        V(u, U) = \sup\left\{ \int_{U} u\div{\phi}\dd{x} : \phi\in[C_{c}^{1}(U,\R)]^{n}, \norm{\phi}_{\infty} \leq 1 \right\}.
%    \]
%    We define the set of functions of \textit{locally bounded variation} to be
%    \[
%        BV_{loc}(\Omega) = \{u\in L^{1}_{loc}(\Omega) : V(u,U)< +\infty \text{ for all } U\Subset \Omega\}.
%    \]
%\end{definition}

%An equivalent, and admittedly more transparent, characterization of $ BV_{loc} $ functions can be given as follows. 
%
%\begin{proposition}[Characterization of $ BV_{loc} $]\label{bvlocchar}
%    Suppose $ u\in BV_{loc}(\Omega) $. Then there exists a Radon measure $ \mu $ on $ \Omega $ and a $ \mu $-measurable $ \sigma:\Omega\to \R^{n} $ with $ | \sigma| = 1$ $ \mu $-a.e. and
%    \[
%        \int_{\Omega} u\div{\phi}\dd{x} = -\int_{\Omega} \phi\cdot \sigma \dd{\mu} \text{ for all } \phi\in C_{c}^{1}(\Omega,\R^{n}).
%    \]
%\end{proposition}
%
%\begin{proof}
%    This is a routine application of the Riesz–Markov–Kakutani representation theorem. To this end, define a linear functional $ L:C_{c}^{1}(\Omega,\R^{n}) \to \R $ by $ L(\phi) = -\int_{\Omega} u\div{\phi}\dd{x} $.
%    
%    For open $ U\Subset \Omega $, the quantity $ c(U):= \sup\{L(\phi) : \phi\in C_{c}^{1}(U,\R^{n}), \norm{\phi}_{\infty} \leq 1\} $ is finite by assumption, whence 
%    \[
%        |L(\phi)| \leq c(U) \norm{\phi}_{\infty} \text{ for all } \phi\in C_{c}^{1}(U,\R^{n}).
%    \]
%    Let $ K\sub \Omega $ be a fixed compact set, and choose open $ U\Subset\Omega $ containing $ K $. Then for $ \phi\in C_{c}(\Omega,\R^{n}) $ with $ \supp(\phi)\sub K $, there exists a sequence $ (\phi_{k})_{k} $ in $ C_{c}^{1}(U,\R^{n}) $ such that $ \phi_{k}\to \phi $ uniformly on $ U $. 
%
%    Define an extension $ \widetilde{L}:C_{c}(\Omega,\R^{n})\to \R $ of $ L $ by $ \widetilde{L}(\phi) = \lim_{k\to\infty}L(\phi_{k}) $, which exists and is well-defined by the above inequality. As $ C_{c}(\Omega, \R^{n}) $ is dense in $ C_{0}(\Omega, \R^{n}) $, continuity guarantees a unique continuous linear extension to the whole of $ C_{0}(\Omega,\R^{n}) $. Applying the Riesz Representation Theorem to this extension gives the conclusion.
%\end{proof}
%

\subsection{Classical Perimeter}\label{sec:perimeter}

As alluded to earlier, the entire discussion around functions of (locally) bounded variation and Gauss-Green formula leads to properties of the boundary of a set. One of the first such properties one becomes intuitively ``familiar'' with is the notion of \textit{perimeter}. From a young age, we are led to believe that perimeter is an ``obvious'' concept.

%TODO insert more history.
It might surprise the reader that the notion of perimeter was not systematically (and definitively) treated until the 1950s. 
Equipped with the Hausdorff measure, one initially jumps to thinking of the perimeter of a set in $ \R^{n} $ as the $ (n-1) $-dimensional Hausdorff measure of its boundary. Indeed, for nice enough boundaries (say $ C^1 $), this definition is good enough. Moreover,  one can show using just the Gauss-Green formula that the ``perimeter'' of a set is the total variation of its indicator function.
\begin{factexercise}
    Let $ E\sub \R^{n} $ be a bounded open set with $ C^{2} $ boundary. Then $ \H^{n-1}(\partial E) = V(\chi_{E},\R^{n}) $.
\end{factexercise}

%TODO insert graphic of square versus square with line


\begin{definition}
    Let $ \Omega\sub \R^{n} $ be an open set and $ E\sub \R^{n} $ measurable. We define the \textit{perimeter of $ E $ in $ \Omega $} to be the quantity
    \[
        Per(E,\Omega):= V( \chi_{E}, \Omega).
    \]
    The set $ E $ is a \textit{Caccioppoli set} or has \textit{locally finite perimeter} if $ Per(E,\Omega) <+\infty $ for every bounded open subset $ \Omega $. Define the \textit{global perimeter of $ E $} to be the quantity $ Per(E,\R^{n}) $.
\end{definition}


To build intuition, suppose $ E $ is open and has both finite measure and finite (global) perimeter in $ \R^{n} $. Then by Proposition \ref{bvchar}, $ \chi_{E}\in BV(\R^{n}) $ and $ Per(E,\R^{n}) = |D \chi_{E}|(\R^{n}) = \int_{\R^{n}}|D \chi_{E}|$. Recall from Section \ref{sec:bv} that the mesaure $ |D \chi_{E}| $ gives us a rigorous way to treat the intuition that the ``gradient'' of $ \chi_{E} $ should a dirac delta at every point of $ \partial E $ and $ 0 $ everywhere else. Moreover, because the distributional gradient $ |D \chi_{E}| $ is a measure, we can ``integrate'' it (that is, take its total mass). Intuitively, integrating dirac deltas on the boundary of a set ``should'' extract the perimeter of that set, so the definition matches our intuition.
\end{document}
