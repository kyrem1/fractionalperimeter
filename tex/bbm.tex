%! TEX root = ../main.tex
\documentclass[../main.tex]{subfiles}


\begin{document}

\section{Detour: the Bourgain-Brezis-Mironescu Formula}\label{sec:bbm}

The main technical tool to prove Theorem \ref{sto1} is the celebrated \textit{Bourgain-Brezis-Mironescu (BBM) Formula}. In \cite{brezis:2001, brezis:2002}, Bourgain, Brezis, and Mironescu sought to gain new characterizations of Sobolev spaces whilst also contributing to the push to define a degree theory for discontinuous maps. The outcropping of this work was a new formula for Sobolev norms which contextualized a large amount of previously interpolation-theoretic literature.


\begin{definition}
From now on, $ (\rho_{i})_{i=1}^{\infty} $ will denote a sequence of \textit{radial mollifiers} in the sense that 
\begin{equation}
    \rho_{i}\in L^{1}_{loc}((0,+\infty)),\quad \rho_{i}\geq 0
\end{equation}
\begin{equation}
    \lim_{i\to\infty}\int_{\delta}^{\infty} \rho_{i}(r)r^{n-1}\dd{r} = 0 \text{ for all } \delta>0 
\end{equation}
\begin{equation}
    \int_{0}^{\infty}\rho_{i}(r)r^{n-1}\dd{r} = 1 \text{ for all } i\in \N.
\end{equation}


\end{definition}
Throughout this section, $ \Omega $ is either a bounded Lipschitz domain or all of $ \R^{n} $.
Define a dimensional constant $ K_{n,p} $ by 
\[
    K_{n,p}:= \int_{S^{n-1}} |e\cdot \sigma|^p \dd{\H^{n-1}(\sigma)}
\]
where $ e\in S^{n-1} $ is arbitrary.\\

The original BBM formula is as follows. 
\begin{theorem}[\cite{brezis:2001,brezis:2002}, BBM Formula]\label{bbm}
    Suppose $ 1< p <+\infty $.For $ u\in L^{1}_{loc}(\Omega) $,
    \[
        \lim_{i\to\infty} \int_{\Omega}\int_{\Omega} \frac{|u(x)-u(y)|^{p}}{|x-y|^{p}} \rho_{i}(|x-y|)\dd{x} \dd{y} = K_{n,p}\int_{\Omega}| Du|^{p} \dd{x}
    \]
    when $ Du\in L^{p}(\Omega) $ and $ K_{n,p} $ is a constant given by . 
\end{theorem}
Note that the above result does not treat the $ p=1 $ case. In their original 2001 paper \cite{brezis:2001}, Bourgain, Brezis, and Mironescu conjectured that their formula holds in the $ p=1 $ case after replacing $ W^{1,1} $ with $ BV $, but they were only able to obtain the following partial result.
\begin{theorem}[{\cite[Cor.~5]{brezis:2001}}]
    For $ u\in L^1(\Omega) $, there exist constants $ C_{1}, C_{2}>0  $ such that 
    \begin{align*}
        C_{1} [u]_{BV(\Omega)} &\leq  \liminf_{i\to\infty} \int_{\Omega}\int_{\Omega} \frac{|u(x)-u(y)|}{|x-y|} \rho_{i}(|x-y|)\dd{x} \dd{y} \\
        &\leq \limsup_{i\to\infty} \int_{\Omega}\int_{\Omega} \frac{|u(x)-u(y)|}{|x-y|} \rho_{i}(|x-y|)\dd{x} \dd{y} \leq C_{2}[u]_{BV(\Omega)}.
    \end{align*}
\end{theorem}
In 2002, Ha\"im Brezis' student Juan D\'avila \cite{davila:2002} answered their conjecture in the affirmative 

\begin{theorem}[{\cite{davila:2002}, BBM-D\'avila Formula}]\label{bbmd}
    Let $ \Omega\sub \R^{n} $ be a bounded Lipschitz domain or all of $ \R^{n} $. Suppose $ u\in BV(\Omega) $. Then
    \[
        \lim_{i\to\infty} \int_{\Omega}\int_{\Omega} \frac{|u(x)-u(y)|}{|x-y|} \rho_{i}(|x-y|)\dd{x} \dd{y} = K_{n,1}\int_{\Omega}| Du|
    \]
    where $ Du $ is the finite Radon measure corresponding to $ u $ and $ \int_{\Omega} |Du| $ denotes the quantity $ |Du |(\Omega) $.
\end{theorem}

We will use Theorem \ref{bbmd} to obtain a characterization of $ BV(\Omega) $-norms in terms of a renormalized limit of $ W^{s,1}(\Omega) $-norms.





%%TODO add exposition on this
%We will outline the proof of Theorem \ref{bbmd}. This proof is the main technical novelty of this paper and we will connect it to asymptotics of fractional perimeter. \\
%
%
%
%%%%%%%%%%%%%%%%%%%%%%%%%%%%%
%
%Consider the Radon measures $ \mu_{i} $ given by  
%
%\[
%    \dd{\mu_{i}} := \lr{\int_{\Omega}\frac{|u(x)-u(y)|}{|x-y|} \rho_{i}(|x-y|) \dd{x}}\dd{y}
%\]
%
%
%Davila proves an extension theorem analagous to %TODO CITE EXTENSION THEOREMS IN APPENDIX
%but with an extra condition that, in addition to the BV norm in $ \R^{n} $ being controlled, we also have control over the BV measure in neighborhoods of $ \partial \Omega $.
%
%\begin{proposition}[\textcite{davila:2002}]
%    Existence of bounded extension operator $ \mc{E}: BV(\Omega)\to BV(\R^{n}) $ for nice $ \Omega\sub \R^{n} $.
%\end{proposition}




%%%%%%%%%%%%%%%%%%%%%%%%%%%%





\end{document}
