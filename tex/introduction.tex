%! TEX root = ../main.tex
\documentclass[../main.tex]{subfiles}


\begin{document}

\section{Introduction}

First introduced by Caffarelli in 2011 \cite{caffarelli:nonlocal, caffarelli:limit}, the study of the fractional perimeter (and more generally nonlocal perimeters) has devloped into a highly active area of research. We mention the 2019 monograph by Maz\'on \cite{mazon:nonlocal} for an overview of the development of the field. 

The quantity known as fractional perimeter initally arose as the energy functional for a nonlocal version of motion by mean curvature. We will give a much more elementary motivation in this paper by appealing to physical intuition.

Recall Newton's law of gravitation which states that the size of the gravitational interaction force between two point masses is inversely proportional to the square of the distance between them. Likewise, Coulomb's law states that the size of the electrostatic force between two point charges is inversely proportional to the square of the distance between them.

In both of these examples, we have some ``interaction force'' between two particles $ \mathbf{r},\mathbf{r^{\prime}} $ given by $ \frac{1}{|\mathbf{r}-\mathbf{r^{\prime}}|^{2}} $. In sense, the power of $ 2 $ in the denominator controls how strongly this force quantity weights distant interactions.


%TODO fix these theorems
\begin{maintheorem}\label{sto1}
    Let $ \Omega\sub \R^{n} $ be a bounded open set with Lipschitz boundary. If $ E\sub \R^{n} $ is a Caccioppoli set, then
    \begin{equation}
        \lim_{s\to1}(1-s) Per_{s}(E, \Omega) = \omega_{n-1} Per(E,\cls{\Omega}).
    \end{equation}
\end{maintheorem}


\begin{maintheorem}\label{sto0}
    Let $ \Omega $ be either a bounded open set with Lipschitz boundary or all of $ \R^{n} $. Let $ E\sub \Omega $ be a Caccioppoli set. Then 
    \begin{equation}
        \lim_{s\to0} sPer_{s}(E,\Omega) = \frac{\omega_{n-1}}{2} \mathcal{L}^{n}(E)
    \end{equation}
\end{maintheorem}

\end{document}
