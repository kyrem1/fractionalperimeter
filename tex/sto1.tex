%! TEX root = ../main.tex
\documentclass[../main.tex]{subfiles}


\begin{document}

\section{Asymptotics of $ P_{s}(E,\Omega) $ as $ s\to 1^{-} $}

\subsection{Local Contribution to $ P_{s}(E,\Omega) $}

In this subsection, we estimate the contribution of $ P_{s}^{L} $ to the $ s $-perimeter as $ s\to1^{-} $. Our main tool is the Bourgain-Brezis-Mironescu-Davila Formula \ref{bbmd}.\\

Recall that $ P_{s}^{L}(E,\Omega) = \frac{1}{2}[ \chi_{E} ]_{W^{s,1}(\Omega)} $. Let $ (s_{i})_{i=1}^{\infty} $ be a sequence of integers with $ s_{i}\to1^{-} $. Choose $ R>>0 $ such that $ \Omega\sub B_{\frac{R}{2}}(0) $. Define mollifiers $ \rho_{i}:(0,+\infty)\to[0,+\infty) $ as in \ref{bbmd} by
\begin{equation}\label{mollifiers}
    \rho_{i}(t):= a_{s_{i}}\frac{1-s_{i}}{t^{n+s_{i}-1}}\chi_{(0,R)}(t)
\end{equation}
Where $ a_{s_{i}} := \frac{1}{\omega_{n-1}R^{1-s_{i}}}$ is chosen such that $ \int_{\R^{n}} \rho_{i}(|x|)\dd{x} = 1 $. Using polar coordinates, one may check that the mollifiers $ (\rho_{i})_{i} $ satisfy the conditions of \ref{bbmd}, namely that
\[
    \text{for all } \delta>0,\quad\lim_{i\to\infty}\int_{\delta}^{\infty} \rho_{i}(r)r^{n-1}\dd{r} = \lim_{i\to\infty} \frac{1}{\omega_{n-1}} \lr{1-\lr{\frac{\delta}{R}}^{1-s_{i}}} = 0.
\]
For $ u\in W^{s,1}(\Omega) $, as $ \Omega-\Omega\sub B_{R}(0) $, we expand 
\begin{align*}
    \int_{\Omega} \int_{\Omega} \frac{|u(x)-u(y)|}{|x-y|} \rho_{i}(|x-y|) \dd{x}\dd{y} &= a_{s_{i}}(1-s_{i})\int_{\Omega}\int_{\Omega} \frac{|u(x)-u(y)|}{|x-y|^{n+s_{i}}}\dd{x}\dd{y} \\
    &= a_{s_{i}}(1-s_{i})[u]_{W^{s_{i},1}(\Omega)}.
\end{align*}
Then for $ u\in BV(\Omega) $,
\begin{align*}
    \lim_{i\to\infty} (1-s_{i}) [u]_{W^{s_{i},1}(\Omega)} &= \lim_{i\to\infty}\frac{1}{a_{s_{i}}}\int_{\Omega} \int_{\Omega} \frac{|u(x)-u(y)|}{|x-y|} \rho_{i}(|x-y|) \dd{x}\dd{y} \\
    &= \omega_{n-1}K_{1,n}[u]_{BV(\Omega)}.
\end{align*}
As the sequence $ (s_{i}) $ was arbitrary, we conclude that $ \lim_{s\to1^{-}}(1-s)[u]_{W^{s,1}(\Omega)} = \omega_{n-1}K_{n,1}[u]_{BV(\Omega)} $. In summary, we have shown the following proposition.

\begin{proposition}\label{localsto1}
    Let $ \Omega\sub \R^{n} $ be a bounded domain with Lipschitz boundary. If $ E\sub \R^{n} $ has finite perimeter inside $ \Omega $, then
    \[
        \lim_{s\to1^{-}}(1-s)P_{s}^{L}(E,\Omega) = \frac{\omega_{n-1}}{2}K_{n,1}Per(E,\Omega).
    \]
\end{proposition}

\subsection{Nonlocal Contribution to $ P_{s}(E,\Omega) $}

The aim of this subsection is to show that the limiting behavior of the nonlocal contribution to $ s $-perimeter, $ P_{s}^{NL}(E,\Omega) $, is controlled by how much the set $ E $ fails to intersect $ \partial \Omega $ ``transversally.'' Namely, we will demonstrate that
\[
    \limsup_{s\to1^{-}}(1-s)P_{s}^{NL}(E,\Omega) \preceq \lim_{\delta\to0^{+}}Per(E,N_{\delta}(\partial \Omega))
\]
where $ N_{\delta}(\partial \Omega) $ denotes the $ \delta $-tubular neighborhood of $ \partial \Omega $. The results and methods in this subsection are new developments due to Lombardini \cite{lombardini:2019}.
%TODO determine constant for this preceq
%TODO insert pictures for failure of transversal intersection


%TODO find the right place for this.




\begin{theorem}[{\cite[Prop.~2.5]{lombardini:2019}}]
    Let $ \Omega\sub \R^{n} $ be a bounded open set with Lipschitz boundary. Suppose that $ E\sub \R^{n} $ has finite perimeter inside $ \Omega $ and $ Per(E,\partial \Omega) = 0 $. Then 
    \[
        \lim_{s\to1^{-}}(1-s) P_{s}^{NL}(E,\Omega) = 0 
    \]
\end{theorem}
%TODO maybe have to assume finite perimeter in a neighborhood of $ \Omega $ (e.g. D_1).

%TODO make graphic accurate (i.e. remove touching of boundaries
\begin{wrapfigure}{r}{0.5\textwidth}
    \begin{center}
    \includesvg[width=0.48\textwidth]{figures/approximation1.svg}
    \end{center}
    \vspace{-10mm}
\end{wrapfigure}


We strictly approximate $ \Omega $ from inside and outside by bounded open sets with Lipschitz boundary, i.e. consider sets $ (A_{i})_{i}, (D_{i})_{i} $ such that
\begin{itemize}
    \item $ A_{i}\sub A_{i+1} \Subset \Omega $ and $ \Omega = \bigcup_{i\in\N}A_{i} $;
    \item $ D_{i}\supseteq D_{i+1}\Supset \cls{\Omega}  $ and $ \cls{\Omega} = \bigcap_{i\in\N}D_{i}$.
\end{itemize}
We make a couple auxiliary definitions.
\begin{itemize}
    \item $ \Omega_{i}^{+} := D_{i}\setminus \cls{\Omega} $, the outer excess portion;
    \item $ \Omega_{i}^{-} := \Omega\setminus \cls{A_{i}} $, the inner excess portion;
    \item $ T_{i}:= \Omega_{i}^{+}\cup \Omega_{i}^{-} \cup \partial \Omega $, the total excess;
    \item $ d_{i}:= \min\{d(A_{i}, \partial \Omega), d(\partial D_{i},  \Omega) \} $ minimum distance from approximation to the boundary of $ \Omega $.
\end{itemize}



\vspace{1.6mm}
%TODO investigate possible typo in d_i
As $ |D \chi_{E}| $ is a finite Radon measure on $ D_{1} $ and all other relevant sets are nested inside $ D_{1} $, continuity from above implies that
\[
    Per(E,\partial \Omega) = |D \chi_{E}|(\cap_{i} T_{i}) = \lim_{i\to\infty}|D \chi_{E}|(T_{i}) = \lim_{i\to\infty}Per(E,T_{i}).
\]
Hence, in order to control the nonlocal contribution ot the $ s $-perimeter in terms of $ Per(E,\partial \Omega) $, it suffices to obtain control in terms of $ Per(E,T_{i}) $ for fixed $ i $. First, recall the two interactions involved in $ P_{s}^{NL} $ are given by
\begin{equation}\label{nlenergydef}
    P_{s}^{NL}(E,\Omega) = L(E\cap \Omega, E^{c}\cap \Omega^{c}) + L(E^{c}\cap \Omega, E\cap \Omega^{c})
\end{equation}
As $ P_{s}^{NL}(E,\Omega) = P_{s}^{NL}(E^{c},\Omega) $ and $ Per(E,T_{i}) = Per(E^{c},T_{i}) $, it suffices to estimate only the first interaction term as an estimate for the second term would follow from this via the replacement $ E\usub E^{c} $. We proceed by decomposing the interaction as 
\begin{equation}\label{decomp1}
    L(E\cap \Omega, E^{c}\setminus\Omega) = L(E\cap \Omega, E^{c}\cap \Omega_{i}^{+}) + L(E\cap \Omega, E^{c}\cap (\Omega^{c}\setminus D_{i})).
\end{equation}
\begin{figure}[H]
    \vspace{-5mm}
    \includesvg[scale=0.75]{figures/interactionOmegaPlusDecompose.svg}
    \vspace{-5mm}
\end{figure}

%TODO insert transition phrase

To estimate these interactions, we need the following claim. %TODO possibly adjust this sentence
\begin{claim}
    If $ y\in \Omega $ and $ \rho_{y}:= \sup\{\rho>0: B_{\rho}(y)\sub D_{i}\} $, then $ \rho_{y}\geq d_{i} $ and thus $ B_{d_{i}}(y)\sub D_{i} $.
\end{claim}

To prove this claim, recall the following fact from point-set topology. 
\begin{fact}\label{distancefact}
    If $ X\sub \R^{n} $ is nonempty and $ y\in \R^{n}\setminus\Int{X} $, then $ d(y,X) = d(y,\partial X) = d(y, \partial X^{c})$.
\end{fact}
This claim follows by taking $ X=D_{i}^{c} $ in Fact \ref{distancefact} and then noting that $ B_{d(y,D_{i}^{c})}(y)\sub D_{i} $ holds by definition.

Now for $ y\in \Omega $, by the following claim we estimate
\begin{equation}\label{intestimate1}
    \int_{\Omega^{c}\setminus D_{i}} \frac{1}{|x-y|^{n+s}}\dd{x} \leq \int_{\R^{n}\setminus B_{d_{i}}(y)}\frac{1}{|x-y|^{n+s}}\dd{x} = \frac{\omega_{n-1}}{s} d_{i}^{-s} .
\end{equation}
Hence, after integrating over $ \Omega $,
\begin{align}\label{energyest1}
    L(E\cap \Omega, E^{c}\cap (\Omega^{c}\setminus D_{i})) \leq L(\Omega, \Omega^{c}\setminus D_{i}) = \int_{\Omega}\int_{\Omega^{c}\setminus D_{i}} \frac{1}{|x-y|^{n+s}}\dd{x}\dd{y} \leq \omega_{n-1}\frac{\mathcal{L}^{n}(\Omega)}{s}d_{i}^{-s} 
\end{align}
giving us a bound on the second term in \ref{decomp1}. To deal with the first term, we use a similar technique as in \ref{decomp1}.  We decompose the interaction with respect to the inner approximation $ A_{i}\nearrow \Omega $ and its inner excess $ \Omega_{i}^{-} $ to obtain
\begin{equation}\label{decomp2}
    L(E\cap \Omega, E^{c}\cap \Omega_{i}^{+}) =  L(E\cap \Omega_{i}^{-}, E^{c}\cap \Omega_{i}^{+}) + L(E\cap A_{i}, E^{c}\cap \Omega_{i}^{+}).
\end{equation}
We analyze the second term in $ \ref{decomp2} $ and use similar logic to the estimate in \ref{intestimate1} to find
\begin{align}\label{energyest2}
    L(E\cap A_{i}, E^{c}\cap \Omega_{i}^{+}) \leq L(A_{i}, \Omega_{i}^{+}) \leq L(A_{i}, \Omega^{c}) = \int_{A_{i}}\int_{\Omega^{c}}\frac{1}{|x-y|^{n+s}}\dd{x}\dd{y} \leq \omega_{n-1}\frac{\mathcal{L}^{n}(A_{i})}{s} d_{i}^{-s}.
\end{align}
To deal with the first interaction term in \ref{decomp2}, we note that both $ \Omega_{i}^{-} $ and $ \Omega_{i}^{+} $ are contained in $ T_{k} $. Hence by definition, we see
\begin{equation}\label{symmetricest}
    L(E\cap \Omega_{i}^{-}, E^{c}\cap \Omega_{i}^{+}) \leq L(E\cap T_{i}, E^{c}\cap T_{i}) = P_{s}^{L}(E, T_{i})
\end{equation}
which, in the context of Proposition \ref{localsto1}, hints towards an asymptotic connection between the energy in question and the classical perimeters of $ E $ relative to the $ T_{i} $s. Now, combining estimates \ref{energyest1}, \ref{energyest2}, and \ref{symmetricest}, we have
\[
    L(E\cap \Omega, E^{c}\cap \Omega^{c}) \leq P_{s}^{L}(E,T_{i}) + 2 \omega_{n-1}\frac{\mathcal{L}^{n}(\Omega)}{s}d_{i}^{-s}.
\]
As discussed previously, the same estimate holds for the other term in \ref{nlenergydef}, so our final energy bound becomes
\begin{equation}\label{finalenergy}
    P_{s}^{NL}(E,\Omega) \leq 2P_{s}^{L}(E,T_{i}) + 4 \omega_{n-1}\frac{\mathcal{L}^{n}(\Omega)}{s}d_{i}^{-s}.
\end{equation}
By Proposition \ref{localsto1}, 
\[
    \lim_{s\to1^{-}}(1-s)P_{s}^{L}(E,T_{i}) = \frac{\omega_{n-1}}{2}K_{n,1}Per(E,T_{i})
\]
hence \ref{finalenergy} implies
\[
    \limsup_{s\to1^{-}}(1-s)P_{s}^{NL}(E,\Omega) \leq \omega_{n-1}K_{n,1}Per(E,T_{i}).
\]
As this holds for all $ i $, it follows that 
\[
    \limsup_{s\to1^{-}}(1-s)P_{s}^{NL}(E,\Omega) \leq \lim_{i\to\infty}\omega_{n-1}K_{n,1}Per(E,T_{i}) = \omega_{n-1}K_{n,1}Per(E,\partial \Omega).
\]
When the RHS is zero, the result follows.











\end{document}
