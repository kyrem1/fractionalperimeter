%! TEX root = ./main.tex
%\documentclass[12pt]{article}
\documentclass[reqno, openany, amssymb, psamsfonts]{amsart}

%--------Packages-------------

\usepackage{wrapfig}
\usepackage{kyrem1sty}

\hypersetup{%
  bookmarksnumbered=true,%
  bookmarks=true,%
  colorlinks=true,%
  linkcolor=blue,%
  citecolor=blue,%
  filecolor=blue,%
  menucolor=blue,%
  pagecolor=blue,%
  urlcolor=blue,%
  pdfnewwindow=true,%
  pdfstartview=FitBH}   
 
\usepackage{mathrsfs,comment}
\usepackage[normalem]{ulem}
\usepackage{url}
\usepackage[all,arc,2cell]{xy}

\usepackage{svg}
%----------------------------



%-----------Misc-------------
\usepackage{setspace}
\setdisplayskipstretch{2}
\numberwithin{equation}{section}
\setcounter{tocdepth}{1}
%----------------------------



%--------Bibliography---------
\usepackage[backend=biber,style=alphabetic]{biblatex}
\addbibresource{fractional.bib}
%----------------------------


%--------Subfiles Setup-------
\usepackage{subfiles}
%----------------------------

%I. Organizational
%TODO:   - Organize Paper
%I. Exposition
%TODO:   - Write Introduction
%TODO:   - Explain why we stop at assuming P(E, \partial \Omega) = 0 and present the further results
%II. Graphics
%TODO:   - indicator function length of gradient plot
%TODO:   - Picture of non transverse nonlocal contribution




%--------Metadata------------
\title{Asymptotics of Fractional Sobolev Norms and $s$-Perimeter}
\author{James Harbour}
%----------------------------


%--------Content-------------
\begin{document}

\maketitle
\begin{abstract}
  In this paper, we survey the notion of fractional $ s $-perimeter and motivate its name by proving an asymptotic relation to the classical perimeter in the sense of De Giorgi. We also prove an asymptotic relation between the $ s $-perimeter and the Lebesgue measure, clarifying the intuition that the $ s $-perimeter is an ``interpolation'' between area and perimeter.

  We develop the necessary geometric and analytic tools from the theories of $ BV $-functions and fractional Sobolev spaces in order to approach the literature on nonlocal functionals such as the $ s $-perimeter.
\end{abstract}

\tableofcontents



\subfile{./tex/introduction.tex}


%\subfile{./tex/preliminarydefs.tex}


\subfile{./tex/bvandcacc.tex}


\subfile{./tex/fracsobolev.tex}


\subfile{./tex/bbm.tex}


\subfile{./tex/fracperim.tex}


%\subfile{./tex/asymptotics.tex}


\subfile{./tex/sto0.tex}


\subfile{./tex/sto1.tex}

\section*{Acknowledgement}

I would like to Peter May for organizing this wonderful program and my REU mentor Adrian Chun-Pong Chu for his guidance and enthusiasm. I would never have touched the wonderful field of geometric measure theory without his guidance.

The UChicago REU showed me just how big math really is, and I will always look back on this past summer as a turning point in my life. Im incredibly grateful to have met lifelong friends and gained new lifelong interests during my time there.

%-----Appendix-----
%\newpage
%\appendix
%
%%\subfile{./tex/appendix/integersobolev.tex}
%
%
%\subfile{./tex/appendix/symmrearr.tex}
%
%\subfile{./tex/miscfra.tex}


%------Biblio------
\newpage
\printbibliography

\end{document}
